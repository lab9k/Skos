% Docker
Skosmos werd opgezet a.d.h.v. Docker. De Dockerfile kan men \href{https://github.com/lab9k/Skos/blob/master/Dockerfile}{hier} vinden. In de volgende paragrafen zullen we een korte samenvatting geven over hoe Docker werkt en kan gebruikt worden.

\subsection{Opzetten in Docker}
Allereerst moet Docker natuurlijk geïnstalleerd worden, we raden tenhartste aan dit op een linux distributie te doen.\footnote{Gebruik deze handige  \href{https://docs.docker.com/install/linux/docker-ce/ubuntu/}{install guide} voor ubuntu.}

Met Docker kan men een \textit{image} die de nodige software en configuratie bevatten voor applicatie. Zo worden alle applicaties en hun dependencies mooi verzameld en afgesloten van de buitenwereld. Een \textbf{image} kan men beschouwen als een \textit{class definition}. De afgeleide instanties van deze klassen noemen we \textbf{containers}.

Deze \href{https://semaphoreci.com/community/tutorials/dockerizing-a-php-application}{tutorial} is alvast een goede inleiding tot Docker. We werkten echter niet met VM's en docker-machine zoals in het eerste deel van de tutorial wordt besproken. 

\subsection{Workflow}
Als men met Docker werkt zal met bij het opzetten van een container meestal de volgende stappen steeds uitvoeren:



% TODO add as appendix + remove old image and container
\begin{minted}{bash}
#!/bin/bash

image_name="kies een naam"
container_name="kies een naam"

# build image
sudo docker build -t $image_name .

# build container
sudo docker run -itd --name=$container_name $image_name

# attach to container with bash
sudo docker exec -it $container_name bash

\end{minted}

\subsection{Dockerfile}
% TODO

% TODO https://github.com/moby/moby/issues/3378

\subsection{Container networking}
% TODO
